\section{Risk-aware Abstraction for\\ Risk-sensitive Control Problem}\label{sec:compute_abs}

%---------- Preliminaries on ABCS -----
%\section{Preliminaries on Abstraction-based Controller Synthesis}\label{sec:prelims_abcs}
\subsection{Preliminaries}\label{sec:prelims_abcs}

The method that we primarily build upon is the so-called Abstraction-Based Controller Synthesis techniques (ABCS).
In the following, we briefly recap one of the several available ABCS techniques.

First, we introduce certain types of finite state abstract transition systems that approximates the continuous system dynamics.\\
\noindent\textbf{Finite State Transition System.}
A \emph{finite state transition system} is a tuple $\Delta= (\Xa, \Ua, \fa)$ that consists of a finite set of states $\Xa$, a finite set of control actions $\Ua$, and a set-valued transition map $\fa:\Xa\times \Ua\setmap \Xa$.

\smallskip
\noindent\textbf{Finite State Abstraction of Control Systems.}
A finite state transition system $\Delta$ is called a \emph{finite state abstraction}, or simply an abstraction, of a given control system $\Sys$ if certain relation holds between the transitions of $\SysT$ and the transitions of $\Delta$.
Depending on the controller synthesis problem at hand, there are several such relations available in the literature. %f\cite{bisimulation-paulo-girard, alt-bisimulation, dist-bisimulation, FRR, augmented transition systems, jun liu rocs}. 
The one that we use in our work is the \emph{feedback refinement relation} (FRR) \cite{ReissigWeberRungger_2017_FRR}.

% In this paper, we do not formally define FRR, nor do we discuss how to find an abstraction that is in FRR with the given control system: A formal treatment can be found in \cite{ReissigWeberRungger_2017_FRR}.
% Instead, we only define FRR informally, and assume that there is an algorithm, called $\proc$, that computes the sought abstraction.
An FRR is a relation $R\subseteq \Xc\times \Xa$ between $\SysT$ and a finite-state transition system $\Delta$, written as $\SysT \preccurlyeq_R \Delta$, so that for every $(\xc,\xa)\in R$, the set of allowed control inputs (element of $\Ua$) from $\xa$ is a subset of the set of allowed control inputs (element of $\Uc$) from $\xc$, and moreover when the same allowed control input is applied to both $\xa$ and $\xc$, the image of the set of possible successors of $\xc$ under $R$ is contained in the set of successors of $\xa$.


%\removeb
%Before defining FRR, we introduce some notation overloading.
%We use the notation $\Uc(\xc)$ and $\Ua(\xa)$ to represent respectively the set of allowed control inputs at any given states $\xc\in \Xc$ and $\xa\in \Xa$ of $\SysT$ and $\Delta$.
%Formally, $\Uc(\xc):=\set{u\in \Uc\mid \exists w\in \Wnor\;.\;\fc(\xc,u,w)\neq \emptyset}$, and $\Ua(\xa):=\set{u\in \Ua\mid \fa(\xa,u)\neq \emptyset}$.
%Also, we will occasionally interpret relations as functions for simpler notation: for any relation $R\subseteq S_1\times S_2$ and for any $s_1\in S_1$, we define $R(s_1):=\set{s_2\in S_2\mid (s_1,s_2)\in R}$, and for any $s_2\in S_2$, we define $R^{-1}(s_2):=\set{s_1\in S_1\mid (s_1,s_2)\in R}$.
%Moreover, for any set $S\subseteq S_1$, we define $R(S):=\cup_{s\in S} R(s)$, and for any $S\subseteq S_2$, we define $R^{-1}(S):=\cup_{s\in S} R^{-1}(s)$.
%
%\begin{definition}
%	Let $\SysT= (\Xc, \Uc, \ft)$ be a sampled-time system, and $\Delta= (\Xa, \Ua, \fa)$ be a finite state transition system s.t.\ $\Ua\subseteq \Uc$.
%	Let $R\subseteq \Xc\times \Xa$ be a relation s.t.\ every $(\xc,\xa)\in R$ satisfies the following conditions:
%	\begin{enumerate}[(i)]
%		\item $\Ua(\xa)\subseteq \Uc(\xc)$, and
%		\item $u\in \Ua(\xa)\Rightarrow \cup_{w\in \Wnor}R(\ft(\xc,u,w))\subseteq \fa(\xa,u)$.
%	\end{enumerate}	
%	Then $R$ is a feedback refinement relation from $\SysT$ to $\Delta$.	
%\end{definition}
%
%We use the notation $\SysT \preccurlyeq_R \Delta$ to denote that $R$ is an FRR from $\SysT$ to $\Delta$, in which case $\Delta$ serves as our finite state abstraction for the control system $\Sys$.
%\removee
Within this paper we assume that there is an algorithm, called $\proc$, which takes as input a given control system $\Sys$ and a given set of additional tuning parameters $P$ (like the state space discretization and the control space discretization), and outputs an abstract finite state transition system $\Delta$ and an associated FRR $R$, s.t.\ $\SysT\preccurlyeq_R \Delta$.
For the actual implementation of $\proc(\Sys,P)$, we refer the reader to \cite{ReissigWeberRungger_2017_FRR}.

Without going into the details of how the parameter set $P$ influences the outcome of $\proc$, we would like to point out one property that we expect to hold.
Let $\SysT=(\Xc,\Uc,\Wnor,\fc)$ and $\SysT'=(\Xc,\Uc,\Wnor',\fc')$ be two different control systems with same state and control input spaces.
Suppose $\Abs=(\Xa,\Ua,\fa)$ and $\Abs'=(\Xa',\Ua',\fa')$ be two transition systems computed using $\proc$ using the same parameter set $P$ i.e.\ $\proc(\SysT,P)=\Abs$ and $\proc(\SysT',P)=\Abs'$.
Then there are one-to-one correspondences between $\Xa$ and $\Xa'$, and between $\Ua$ and $\Ua'$.
Moreover, there is an FRR $R$ so that both $\SysT\preccurlyeq_R \Abs$ and $\SysT'\preccurlyeq_R \Abs'$ hold.
We will abuse this property, and will use the same state space and input space for both $\Abs$ and $\Abs'$.

%\removeb
%
%\smallskip
%\noindent\textbf{Abstraction-based Controller Synthesis.}
%Abstraction-based Controller Synthesis (ABCS) is a procedure that takes as input the description of a control system $\Sys$ and a control specification $\mathit{Win}$ in Linear Temporal Logic (LTL), and automatically produces a sound feedback controller for $\SysT$ as output.
%The algorithms in ABCS generally work in three stages: First, a finite state abstraction of the given continuous control system is computed.
%In our case, this is done using the procedure $\proc$.
%Second, the finite state abstraction $\Delta$ produced by $\proc$ is used to synthesize a discrete control strategy $\widehat{C}:\Xa\rightarrow \Ua$ by using well-known techniques from reactive synthesis literature \cite{xx}. %\cite{maler pnueli, gr1}.
%In the end, this discrete control strategy is refined into a continuous controller $C:\Xc\rightarrow \Uc$ for $\SysT$ with the help of the FRR $R$ produced by $\proc$ as a by-product.
%
%One of the important advantages that ABCS offers is that $C$ is by design sound: It can be shown that the closed loop $C\parallel\SysT$ guarantees that the specification $\mathit{Win}$ will be satisfied at all instances in all runs of $\SysT$ starting at any state in $\dom{C}$.
%This formal soundness guarantee stems from the defining properties of FRR.
%
%On the downside, the soundness guarantee of the controller $C$ is rigidly conditioned on the correctness of the used control system model $\Sys$.
%For example, if the controlled closed loop experiences a higher disturbance value than the one which was considered during the design phase, then $\mathit{Win}$ could be violated.
%In fact, the existing ABCS techniques do not offer any means to synthesize controllers which are robust against unaccounted for disturbances.
%Addressing this limitation is the subject of this paper.
%
%\removee
%
%\AKS{I would like to have sec IV as a prelim section inside V}


\subsection{Risk-aware Abstraction}

To take into account the effect of occasional occurrences of disturbance spikes in the control system $\SysT$, we introduce a special finite state abstract transition system, called the \emph{bimodal transition system}, with two distinct transition relations.
A bimodal transition system is a tuple $(\Xa,\Ua,\fanor,\fadist)$, where $\Xa$ is a finite set of \emph{abstract states}, $\Ua$ is the finite set of control actions, $\fanor:\Xa\times \Ua\setmap \Xa$ is a set-valued map representing the \emph{normal transitions}, and $\fadist:\Xa\times \Ua\setmap \Xa$ is another set-valued map representing a set of \emph{disturbance transitions}.

\begin{definition}\label{def:risk-aware abstraction}
	Let $(\SysT,\Whi,\Spec)$ be a risk-sensitive controller synthesis problem.
	A bimodal transition system $\Gamma=(\Xa,\Ua,\fanor,\fadist)$ is called a \emph{risk-aware abstraction} of $\SysT$ if there exists a relation $R\subset \Xc\times \Xa$ s.t.\ the following conditions are satisfied:
	\begin{itemize}
		\item $\SysT \preccurlyeq_R (\Xa,\Ua,\fanor)$, and
		\item for all $(\xc,\xa)\in R$, $u\in \Ua(\xa)\Rightarrow \cup_{w\in \Whi}R(\ft(\xc,u,w))\subseteq \fanor(\xa,u)\cup \fadist(\xa,u)$,
	\end{itemize}
	where $\Ua(\xa)$ is the set of control inputs allowed in the state $\xa$, i.e.\ $\Ua(\xa):=\set{u\in\Ua\mid \fanor(\xa,u)\neq \emptyset}$.
\end{definition}

\REFalg{alg:compute risk-aware abstraction} computes risk-aware abstraction for a given risk-sensitive controller synthesis problem and a given parameter set $P$.
Note that, \REFalg{alg:compute risk-aware abstraction} can be implemented symbolically.

\begin{algorithm}
	\caption{$\procRisk$}
	\label{alg:compute risk-aware abstraction}
	\begin{algorithmic}[1]
		\INPUT $(\SysT,\Whi,\Spec)$, some parameter set $P$ for $\proc$
		\OUTPUT $\Gamma=(\Xa,\Ua,\fanor,\fadist)$ 
		\State Compute $\Abs^{\mathrm{nor}}=(\Xa,\Ua,\fanor)\gets\proc(\SysT,P)$
		\State Compute $\Abs^{\mathrm{high}}=(\Xa,\Ua,\fa^{\mathrm{high}})\gets\proc\left((\Xc,\Uc,\Whi,\fc),P\right)$
		\State For all $\xa\in \Xa$ and $\ua\in \Ua$, define $\fadist:(\xa,\ua)\mapsto \fa^{\mathrm{high}}(\xa,\ua)\setminus \fanor(\xa,\ua)$
		\State \Return $\Gamma=(\Xa,\Ua,\fanor,\fadist)$
	\end{algorithmic}
\end{algorithm}

% \begin{proposition}
% 	\REFalg{alg:compute risk-aware abstraction} is sound.
% \end{proposition}
% 
% \begin{proof}
% 	The first bullet point in \REFdef{def:risk-aware abstraction} holds naturally due to the property of $\proc$.
% 	Now, because $R$ is also an FRR between $(\Xc,\Uc,\Whi,\fc)$ and $\Abs^{\mathrm{high}}$, hence by definition of FRR \cite{ReissigWeberRungger_2017_FRR}, for all $(\xc,\xa)\in R$, $u\in \Ua(\xa)$ implies that $\cup_{w\in \Whi}R(\ft(\xc,u,w))\subseteq \fanor(\xa,u)\cup \fadist(\xa,u) = \fadist(\xa,\ua)$.
% \end{proof}
